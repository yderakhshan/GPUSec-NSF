\label{sec:motivating}
In this section, we discuss two motivating examples drawn from recently
published timing attacks on GPU code.
%
We will refer to these examples throughout the proposal and, in Thrust 3,
propose to reproduce the attacks and show that our proposed work mitigates
them.

\subsection{AES Encryption}
Advanced Encryption Standard (AES) is one of the most popular encryption algorithms.
%
Given a cipher key, the AES algorithm applies iterative rounds of transformations  to encrypt a confidential input (plaintext) such as passwords and private messages into the ciphertext.
%
Each round constructs a new {\em round key}, and all round keys are considered
to be high-security as any one is sufficient to reconstruct the original
secret.
%
In each round, a transformation is applied to 16-byte blocks of the round key.
%
In order to speed up encryption, many implementations of AES implement the
transformations partially using a lookuptable: each byte of the block is
looked up in the table to obtain a transformed byte, and then an additional
operation is performed on the transformed bytes within the block.
%
Because blocks can be encrypted independently and the same lookup table is used
for all of the transformations in a round, it is logical to parallelize
encryption by having each thread transform a block, and to store the lookup
table in shared memory.

Because the table is stored in shared memory and is indexed using bytes of the
round key, the number of bank conflicts can leak information about the key.
%
Jiang et al.~\cite{AESattack} present such an attack on a table-based
implementation of AES, and show that they are able to reconstruct the
last-round key solely by observing the encryption time.

\subsection{Large Language Models}
\stefan{fill in}
