\subsection{Motivating Example}
\farzaneh{merge with Section 3.}
% \farzaneh{move to the intro if needed: GPU memory is split into several regions, onchip and off-chip. On-chip memory consists of registers,
% caches, and shared memory.
% Off-chip memory
% is GDDR SGRAM, which is logically distributed into
% texture memory, constant memory, local memory, and
% global memory.
% Both texture memory and constant memory are read-only during the GPU kernel execution. Local memory is private to each thread and thus does not play a role in any of the two attack models. Therefore, in this proposal we focus on registers, shared memory, and global memory.
% }
% \farzaneh{Merge the paragraph here or add to the first thrust:
% The victim sends its plaintext to the server through a secure channel that the attacker cannot observe.
% % 
% The server, then, uses a private cipher key to encrypt the plaintext into a ciphertext that is public and, therefore, visible to the attacker.
% %
% The attacker, using the same server for a different purpose, aims to get access to either the victim's plaintext or the cipher key used by the server;  accessing either of these would allow the attacker to decrypt the plaintext, as it can observe the public ciphertext.
% % 
% Even accessing a few words of plaintext or the cipher key is regarded as a security leak, as it allows the attacker to execute a faster brute-force attack on the remaining data.
% }
% \farzaneh{Amdahl's law + GPUs are expensive -> one way is to share, e.g., MPS}

In this thrust, we consider the scenario in which both the victim and attacker are clients of a shared GPU (see Figure~\ref{fig:th2-attack}). 
%
Several prior sudies~\cite{lee} showed that if the attacker's application runs on the same GPU as the victim's, it can potentially access private information left on the GPU memory by the victim's application.
%
This potential leak occurs both in a sequential time-sharing setting and in a concurrent setting, e.g., enabled by NVIDIA Multi-Process Service (MPS).
%
In a sequential setting, multiple GPU processes are executed in an interleaved fashion, with each process having exclusive access to GPU resources for a defined duration. The kernels of these processes are scheduled to run one after the other.
%
In the concurrent setting, multiple GPU processes can share GPU resources simultaneously, enabling multiple kernels to run at the same time.
%
In both cases, a lack of proper memory isolation may lead to an overlap between the memory locations that the attacker can access through normal operations—without any special privileges—and those used by the victim.
%
%
% If the attacker's application runs concurrently with the victim's, it can potentially access private information that is left on the GPU memory by the victim's application.
%
% if the memory locations allocated to the attacker overlap with those used by the victim.
%
% To achieve this, the attacker only needs to access the GPU memory through regular allocation, deallocation, and read/write operations without requiring any special privileges.
%
The details of the attack vary depending on where in the memory the victim's information is located, e.g., in the shared or global memory, and whether the GPU server is concurrent or sequential.
%

\paragraph{Time-sharing setting.}

Figure~\ref{fig:th2-mot}(a), for example, demonstrates an attack on a sequential server with private information stored in shared memory.
%
Similar to Thrust 1, the victim in this scenario is an AES application.
% 
GPU computing models discourage long-running GPU kernels. 
%
Thereby, the AES application uses several kernels, instead of a long-running one, repeatedly and processes the intermediate results. 
%
To increase efficieny, the application keeps its frequently accessed data, e.g., the encryption key, in shared memory, even after a single kernel terminates.
%
This means that if an attacker's kernel is scheduled to run between two kernels of the victim, it can compromise the security of the plaintext. 
%
The attacker can compromise the security by either reading the data left in shared memory, eventually relaying it back to the attacker's CPU (through either a direct or indirect leak), or by overwriting the data, altering the key.
%
In both cases, the attacker knows the key and can infer the plaintext by observing the public cyphertext. 
%
In the literature, this attack is often called the End of Kernel (EOK) attack.


Figure~\ref{fig:th2-mot}(b) demonstrates a similar attack for a sequential server, but this time with the private information stored in global memory.
%
A segment of global memory is allocated to each CUDA context,  and an interleaved attacker kernel cannot access it until the application has terminated and the memory is deallocated.
%
After the application terminates, it deallocates the memory,  which then becomes available for future allocations.
%
However, prior work~\cite{pietro2016TECS} have demonstrated that data can persist even after deallocation, allowing an attacker kernel that runs immediately after the application to access the memory segment with its prior data. 
%
As such, even if the programmer attempts to clear the memory before exiting, this access can still result in information leakage. 
%
For example, in Figure~\ref{fig:th2-mot}(b), the attacker can access the plaintext stored in the global memory by the victim even after the deallocation of the global memory segment by the victim.

\paragraph{Spatial-sharing setting.}

When it comes to concurrent access, the situation becomes more complex as resources are allocated dynamically, and two applications can, in practice, access one segment of the memory simultaneously (see Figure~\ref{fig:th2-mot}(c)).
%
As stated in NVIDIA's MPS manual (Section 2.3.3.1), ``An out-of-range read in a CUDA kernel can access CUDA-accessible memory modified by another process and will not trigger an error, leading to undefined behavior.'' \cite{hayes2017usenix}
\farzaneh{I'm not aware of many prior work on such attacks in MPS}



% Processes sharing the GPU send their requests to the MPS, where these requests are queued and processed by the GPU using a FIFO scheduling policy.
% 
% MPS is designed to enhance performance when a single application process underutilizes the GPU's compute capacity and thus the GPU is shared between multiple processes.
% 
% The operating system schedules the task based on the scheduling policy. 
% GPU computing models discourage long-running GPU kernels because current GPUs do not support preemptive scheduling. Long-running GPU programs thereby use either several kernels or the same kernel repeatedly and process the intermediate results. The main target of this attack are frequently accessed data stored in the per-CU local and per-PE private memory. For example it can load secret keys, AES S-Boc and ... in the local and private memories at the end of each kernel execution. An attacker can easily read this data.
% %
% The memory allocated by any of the client processes is accessible to all other client processes, which compromises memory isolation between them.
%

\subsection{Prior work}
\farzaneh{move to this to the separate section on prior work?}
Prior work has explored various approaches to prevent these vulnerabilities, taking into account both the private information stored in shared memory and that in global memory.
%
Pietro et al.~\cite{pietro2016TECS} proposed to rewrite the code such that it zeroizes all leftover data in shared memory after kernel execution terminates.
%
They demonstrated this approach with a sample program that calculates the sum of two vectors.
%
Additionally, they implemented a zeroizing function within the CUDA runtime to clear all global memory after the context ends.
%
In their experiments, they found that the overhead of zeroizing shared memory was negligible for the sample program.
%
However, the overhead of zeroizing global memory through the CUDA runtime was not negligible.
%
Even if the overhead for zeroizing might be low, their approach prevents low-security data from remaining in shared memory when necessary; it forces the user to transfer the low-security data between CPU and GPU for each kernel execution, disrupting performance optimizations related to data locality and defeating the purpose of GPU programming.
%

%
% Furthermore, their approach toward zeroizing shared memory is somewhat ad-hoc, relying completely on the programmer to rewrite the program to zeroize the shared memory location.
% %

% The overhead of zeroizing might be low (how much?) but still it doesn't allow low security ones to remain in the shared memory, when one needed.
%
Hayes et al.~\cite{hayes2017usenix}, introduced a dynamic taint tracking approach as a complementary measure to identify memory locations tainted with sensitive data.
%
They then use that taint information either to trigger an alarm whenever there is an aunauthorized access to such tainted memory or to zeroize the tainted locations in shared (resp. global) memory after the kernel (resp. context) terminates.
%
Dynamic taint tracking is known to be an expensive procedure. 
%
Hayes et al. optimized it by leveraging GPU-specific features, such as eliminating certain instructions working with runtime parameters and constants from taint tracking, as well as utilizing the large GPU register file.
%
These optimizations significantly reduced the overhead compared to CPU-based taint-tracking algorithms,  achieving a 5-20x speedup across a range of applications.
%
Despite these improvements, the approach still incurs a 5\%-13\% performance penalty, particularly when dealing with shared memory.

% However, none of the prior work address the integrity aspect of the attack, where an attacker could leave malicious data behind in shared memory, potentially compromising the system’s security.

None of the prior work considers that low-integrity writes can be as malicious to security as low-security reads.
%
This is especially concerning when dealing with an active attacker on the GPU, who can injects data into memory to force leaking sensitive information,  e.g., an attacker can rewrite the cipherkey behind in shared memory and observe the resulting ciphertext.


\subsection{Proposed work}
This thrust focuses on addressing EoK and EoC attacks as described in Section~\ref{sec:motivating}.
%
% We provide a brief overview of the main challenges in mitigating such attacks.
%
We propose designing a flow-sensitive information flow control system to capture the taint level propagation by running each warp.
%
A flow-sensitive system allows the confidentiality level,
i.e., security type, of variables and locations to change while running or typing the program. 
%
For example, after each high-security write on a particular location, the security level associated to the location increases to at least the security level of the secret.
%
There are two main challenges in addressing such attacks.


\paragraph{Challenge 1: Fine-grained security annotations for indexed arrays.}
%
We are interested in a fine-grained security annotations to shared and global memory locations, which are usually modeled as arrays or matrices.
%
Prior work~\cite{?} assign a single security level to the whole array, i.e., after a single high-security write on a single index of an array, they change the security type of the entire array to high.
%
While this results in a sound type system, it is too coarse-grained for our application, for which we are interested in identifying, as exact as possible, the locations of the array which may be influenced by a secret. 


\paragraph{Challenge 2: Non-deterministic interleaving of warps and blocks.}
To address the EoK and EoC attacks we need to consider the interleaving between multiple warps within a block (which share the shared memory) and multiple blocks within a kernel (which share global memory).
% %
% This imposes a challenge, as the warps withing a block and blocks within a kernel can be interleaved by a scheduler.
%
The interleaving of warps and blocks is controlled by a scheduler, the exact details of which are not fully known due to the proprietary nature of NVIDIA's implementation and the dependence on runtime events.
% 
As a result, we must assume that any interleaving of warps and blocks is valid.
% 
We model the system with a non-deterministic scheduler and aim to prove that, for any possible interleaving of warps and blocks, the program is secure.

We propose two approaches to address the above challenges.
%
In Task 2.a, we propose an approach based on static information flow control analysis.
%
We design a flow-sensitive taint tracking system that captures the flow of sensitive data throughout the program and memory locations statically via taint levels.
%
While this static approach is low-cost, it only provides an over-approximation of the security guarantees. 
% 
For example, when two warps both access the same shared memory location—one for reading and the other for writing—the static analysis must assume that if the write is high-security, the read is considered tainted. 
% 
However, in practice, if the read always happens before the write, there may be no need to taint the data, as no leak would occur in such a case.

In Task 2.b, we address the limitations of the static analysis approach by incorporating a hybrid method that combines both static and dynamic taint tracking.
% 
% Dynamic taint tracking is inherently expensive, especially if we track more than just a single bit for each memory location.
% Rather than a binary secure/insecure flag, we aim to capture the full spectrum of information flow within the application.
%
To achieve this, we propose using static analysis for each warp (with respect to shared memory) and each block (with respect to global memory) to establish the sensitive flow of information in these units.
%
Then we design a dynamic monitoring to track the actual interleaving of warps and blocks at runtime.
% 
The dynamic taint tracking hardware monitor should detect potential insecure memory accesses, either preventing them or raising an alarm.
% 
Additionally, this monitor can sanitize sensitive (tainted) data objects at the end of their lifecycle.

%
% We plan to use static analysis for each warp and each block and then capture the interleaving using the dynamic monitor.

In Task 2.c, we plan to implement a Taint Analysis Framework based on the ideas of Tasks 2.a and 2.b using OCAML.
\farzaneh{I'm not sure about this.}
\farzaneh{Should we only focus on the shared memory?}
In the proposal we describe our ideas based on the EoK attacks in the time-sharing setting. 
% In thrust 1, we propose an approach based on a static information flow control analysis in which we control the flow.
% %
% It is similar to the previous task, except that we need to consider multiple warps in a block that share the shared memory and multiple blocks in the kernel that share the global memory. 
% %
% The warp schedulers draw from a pool of available/ready warps, and select one or more instruction, per cycle, from each warp, per schedule
% %
% Any possible interleaving of blocks should be valid. presumed to run to completion without pre-emption can run in any order. can run concurrently OR sequentially
% Threads are assigned to Streaming
% Multiprocessors (SM) in block granularity
% Each Block is executed as 32-thread Warps
% , we need to put severe restrictions on the programs, restricting them to store any sensitive information in either shared or global memory. 
% % 
% We, instead, propose a dynamic taint tracking hardware monitor to catch such potential insecure memory accesses, either prevent them or raise an alarm, and changes the route of scheduling, and clears sensitive (tainted) data objects at the end of their life.
%
% The taintable sources are program inputs
% %
% In Thrust 2, we propse a dynamic taint tracking system, considering the expensive.
% %
% Dynamic taint tracking is expensive, particularly if we don't want to consider a single bit and do genreic taint tracking, i.e., instead of only one bit of secure/insecure, capture the full lattice of applications.
% % 
% In prior work they considered making it better in performance by considering things specific to GPU, for example specifying that tid or constant is never tainted.
% %
% We propose a new approach based on combining static analysis and dynamic taint tracking.
% %
% For the static analysis we can capture the flow of information inside the victim's application kernel.
% %
% With this we know exactly which parts are tainted and which parts are not.


% Programmers can manually erase global memory before program exit, but registers and local memory are allocated by the compiler and cannot be as easily cleared.

% optimize the data locality
% minimize data trasnfer between CPU and GPU and between peer GPUs.
% use shared memory for data frequenlty used within SM.
% This severly affects performance and defeats the purpose of GPU programming. 
% %


% optimize the data locality
% minimize data trasnfer between CPU and GPU and between peer GPUs.
% use shared memory for data frequenlty used within SM.

% the warp schedulers draw from a pool of available/ready warps, and select one or more instruction, per cycle, from each warp, per schedule

% Any possible interleaving of blocks should be valid. presumed to run to completion without pre-emption can run in any order. can run concurrently OR sequentially
% Threads are assigned to Streaming
% Multiprocessors (SM) in block granularity
% Each Block is executed as 32-thread Warps

% within a block, threads share data via
% shared memory
% Data is not visible to threads in other blocks




% We propose two approaches to preven this kind of attack.
% %
% The first approach is a static one based on a static type analysis , we need to put severe restrictions on the programs, restricting them to store any sensitive information in either shared or global memory. 
% % 
% This severly affects performance and defeats the purpose of GPU programming. 
% %
% Particularly, when we cannot rely on the attacker's programs to be statically typed.
% %
% We, instead, propose a dynamic taint tracking hardware monitor to catch such potential insecure memory accesses, either prevent them or raise an alarm, and changes the route of scheduling, and clears sensitive (tainted) data objects at the end of their life.
% %
% % The taintable sources are program inputs
% % %

% Dynamic taint tracking is expensive, particularly if we don't want to consider a single bit and do genreic taint tracking, i.e., instead of only one bit of secure/insecure, capture the full lattice of applications.
% % 
% In prior work they considered making it better in performance by considering things specific to GPU, for example specifying that tid or constant is never tainted.
% %
% We propose a new approach based on combining static analysis and dynamic taint tracking.
% %
% For the static analysis we can capture the flow of information inside the victim's application kernel.
% %
% With this we know exactly which parts are tainted and which parts are not.
% Since the applications are scheduled nondeterministically though, and can be used by different clients, we cannot use static taint tracking for interleaving applications in an efficient way. Otherwsise it will be too restrictive.
% %
% We built upon the static information flow control built prior.
% This happens when both kernels are accessing the shared memory, and one is safe while the other one is not. written in shared, is already given by the first thrust.

% It enables data protection that clears sensitive (tainted) data objects at the end of their life range as well as detects leak of the sensitive data in the midst of program execution.


% Examples include face recognition photos, a plain-text message, and encryption key.
% It enables data protection that clears sensitive (tainted) data objects at the end of their life range as well as detects leak of the sensitive data in the midst of program execution.


% %
% The Multi-Process Service (MPS) is a software solution designed to efficiently manage multiple processes sharing a GPU allowing to run them concurrently~\cite{anasic2014CAN, NVDIA2013, li2011ICPP}.
% %
% % It is designed to enhance performance when a single application process underutilizes the GPU's compute capacity.
% %
% Processes sharing the GPU send their requests to the MPS, where these requests are queued and processed by the GPU using a FIFO scheduling policy.
% % 
% MPS is designed to enhance performance when a single application process underutilizes the GPU's compute capacity and thus the GPU is shared between multiple processes.
% %
% However, one of its key limitations is that memory allocated by any of the client processes is accessible to all other client processes, which compromises memory isolation between them.
% %
% % New requests can be accepted even while another application is currently executing a kernel on the GPU.


\subsection{Proposed work}
\subsubsection{Task 2.a.} 
% We first focus on warps/shared memory.
% %
% We will address the challenges with respect to global memory next.
%
We propose a flow-sensitive information flow control type system a la Hunt and Sands\cite{hunt2006popl}  to capture the taint level propagation through a program with multiple warps
%
A flow-sensitive type system allows adjusting the security levels of memory locations while typing the program.
%
For example, after writing data on a particular location, the security level associated with the location increases to at least the security level of the data.
%

To address the first challenge, we associate a security level to each location in shared memory and global memory, and annotate the typing judgment with the warp id ($\mathsf{wID}$), that identifies a specific warp under consideration, and a predicate ($\mathsf{mask}$), that provides an overapproximation of which threads in the warp are executing within a particular branch. 
% 
With this information, our type system tracks changes in security levels at as fine a granularity  as possible.
% %
% For example, if there is a high-secrecy write on the shared memory indexed by a constant, our type system only changes the security annotation of that particular location.
% %
For example, if there is a high-secrecy write on the shared memory indexed by $\mathsf{tid}$, our type system only changes the security annotation of those shared memory locations that are running inside the current warp and satisfy the condition provided by the current mask.


To address the second challenge regarding the interleaving of warps and blocks, we propose collecting an over-approximation of the read and write memory footprints for each warp and ensuring that these accesses do not result in undefined behavior due to a data race.
%
Given the unique parallelization paradigm of GPUs, where all warps run the same code and can be interleaved until reaching synchronization points (indicated in CUDA with the \texttt{syncthreads} instruction) \stefan{Changed formatting of syncthreads}, we can structure the program analysis to track and handle these interleavings effectively.
As such, we can separate the code into several segments, separated by synchronization points.
%
We ensure that synchronization points are placed outside any conditional branches, allowing for precise segmentation.\footnote{It is an error in CUDA for a
\texttt{syncthreads} instruction to be executed inside a conditional branch or loop unless all threads in the warp are active.}
%
The warps can be interleaved freely within a segment but must synchronize before moving to the next segment.
We propose breaking the code into multiple segments, separated by synchronization points.
%
We then focus on the memory footprints of each segment for each warp. 
%
In particular, we track the read and write memory accesses for each segment and warp and ensure that write footprints do not overlap with the read or write footprints of other warps within the same segment. 
%
Read footprints can overlap since concurrent reads do not cause data races.
% The red annotations in the above rules show show we collect such footprints.
% we need to combine the static typing judgments  for each individual warp into an analysis that accounts for how warps interact during execution.
%


%

%
% The core does fast switching between warps when warp has to wait for data for example.
%
% This can be a source for data race, which results in an undefined behavior.
%
% This can comlicate the dynamic taint tracking analysis when there are two accesses to a single location and at least one of them is a write.
%
% One reason is for the simple fact that in the presence of data race all possible cimbinations are available.
%
% The other reason which complicates it even further is that CUDA has weak memory model, meaning that we cannot rely on the sequential consistency in these cases either.
%
% As such, in this task, we propose collecting an over-approximation of footprints for each warp and ensuring that they do not result in an underfined behavior.
%
% For each segment, we make sure that the write footprints of each warp does not overlap with the read and write footprint of the others; two read footprints can overlap.
% %
% The red annotations in the above rules show show we collect such footprints.


% We generalize the judgments for statements to be of the form
% %
% $\Psi; \mathsf{wID};\mathsf{mask};  \Sigma; \pc\vdash s \dashv \Sigma'$.
% %

To formally model the above ideas, we generalize the judgments for statements to be of the form
$\mathsf{wID}; \mathsf{mask};\Sigma; \pc \vdash s  \dashv \Sigma'; \delta_\mathsf{write}; \delta_\mathsf{read}$.
%
Here $\Sigma$ and $\Sigma'$ are the pre- and post-context, respectively, mapping local variables, shared memory, and global memory locations to security levels.
%
They correspond, respectively, to the security levels before and after the execution of the statement $s$ by threads that are in warp $\mathsf{wID}$ and satisfy $\mathsf{mask}$.
The mask $\mathsf{mask}$ indicates which threads within the warp are active.
The $\pc$ represents the usual ``program counter'' level and serves to capture the indirect information flows.
Sets $\delta_\mathsf{write}$ and  $\delta_{\mathsf{read}}$ collect the locations in shared memory that might be accessed during the execution of $s$ by writes and reads, respectively.
%
For expressions, we propose the  typing judgment $\mathsf{wID}; \mathsf{mask};\Sigma \vdash e: \tau; \delta_{\mathsf{read}}$.  Expressions cannot write to any memory locations and thus we do not need to adjust the security level of  locations in a post context; $\tau$ is the security level associated with $e$ in the context $\Sigma$ and $\delta_{\mathsf{read}}$ denotes the read footprint of $e$.


%  variables and concrete levels of the lattice $\mathit{Var} \rightarrow \Psi$.
%

% by threads that satisfy $\mathsf{mask}$ in warp $\mathsf{wID}$, respectively.
%
% We assume that the types remain the same before and after, but the security levels can change by writng on them.
%
% 
% We also assume a lattice $\Psi$.
% %
% We fix one lattice and drop $\Psi$ from the judgments for clarity.
%
%

% To account for the low-integrity writes, we consider a taint level consisting of a pair $\langle c, i \rangle$, where $c$ is the security level and $i$ is the integrity level.
%
% Each judgment comes with a security level $\mathsf{sec}$, which is the level of security of the kernel, this corresponds to the amount of secret that they are allowed to know.
%

% We provide a more fine-grained security for each element of shared memory.
%
Next, we propose a few sample rules capturing the above discussion in more detail. 
%
We start with a simple rule in which the code writes on a shared memory location indexed by a constant and all threads are active ($\mathsf{mask}$ is set to $\mathsf{true}$).
%
{\small\begin{mathpar}
    \infer*[Right=]{
        \mathsf{wID}; \mathsf{true};\Sigma \vdash  e: \iota; \delta_{\mathsf{read}}
    \\
    c\,\, \mathit{is}\, \mathit{a \, constant}
     }{\mathsf{wID}; \mathsf{true};\Sigma, \pc \vdash S[c]\leftarrow e \dashv \Sigma[S[c]\mapsto \pc \sqcup \iota]; \{S[c]\} ;\delta_{\mathsf{read}}}
    \end{mathpar}}
The above rule states that writing an expression $e$ of security level $\iota$ to a shared memory location indexed by a constant $c$ updates the security level of $S[c]$ to $\mathsf{pc} \sqcup \iota$, where $\sqcup$ is the lattice join operator, without affecting the security annotations of other shared memory locations.
% 
Moreover, the write footprint in the conclusion is the singleton set containing the location $S[c]$ since only $S[c]$ is being written to. The read footprint is the same as $\delta_{\mathsf{read}}$, representing the memory locations accessed during the evaluation of $e$. 

% Here, we assume that $e$ does not reference any shared memory locations.
% \farzaneh{Does these rules make sense to you? What if shared memory is dynamically allocated?}

In a more general setting, where the index of the shared memory is not a constant, the static rule may not precisely determine the exact location being written to. 
% 
However, we can provide a conservative overapproximation of the locations being written to by leveraging the warp ID ($\mathsf{wID}$) and $\mathsf{mask}$.
{\small\begin{mathpar}
    \infer*[Right=]{ \mathsf{wID}; \mathsf{mask};\Sigma \vdash e: \iota; \delta_{\mathsf{read}}\\
    S[i]@\tau \in \Sigma
    % x= \bigsqcup_{i \in \mathsf{wID}\cup \mathsf{mask}}x_i
     }{\mathsf{wID}; \mathsf{mask};\Sigma; \pc \vdash S[o]\leftarrow e \dashv \Sigma[S[i]\mapsto \pc \sqcup \iota]_{i@\mathsf{MustW} \in \mathsf{set}(o, \mathsf{wID}, \mathsf{mask})}; \quad \mathsf{set}(o, \mathsf{wID}, \mathsf{mask}) ;\delta_{\mathsf{read}}\\ \qquad\qquad\quad\;\,[S[i]\mapsto (\pc \sqcup \iota)\sqcup \tau_i]_{i@\mathsf{MayW} \in \mathsf{set}(o, \mathsf{wID}, \mathsf{mask})}}
    \end{mathpar}}
% 
The rule above types writing an expression $e$ of security level $\iota$ to a shared memory location indexed by an expression $o$.
%
Here, we assume that $o$ does not reference any shared memory locations, and thus, its read footprint is empty.
%
We construct  $\mathsf{set}(o, \mathsf{wID}, \mathsf{mask})$ as a set of all indices that may be referenced by $o$, considering that any occurrence of $\mathsf{tid}$ in $o$ is restricted to the warp $\mathsf{wID}$ and the active threads indicated by $\mathsf{mask}$. 
%
For example, $\mathsf{set}(\mathsf{tid}, 2, \mathsf{true})$ consists of indices $32-63$.
%
In this particular example, we can construct the set of all written indices exactly.
%
However, in general, some elements of the set are included only as a conservative approximate.
%
To distinguish between locations that are definitely written to and those that may be written to, we introduce two tags, $\mathsf{MustW}$ and $\mathsf{MayW}$, respectively.
%
For each index $i$ in $\mathsf{set}(o, \mathsf{wID}, \mathsf{mask})$, the rule updates the security level based on the tag associated with $i$.
%
If $i$ is a must-write ($\mathsf{MustW}$), it updates the security level to $\mathsf{pc} \sqcup \iota$, and if it is a may-write ($\mathsf{MayW}$), it updates the security level to $(\mathsf{pc} \sqcup \iota) \sqcup \tau_i$, where $\tau_i$ is the security level of $S[i]$ before running the statement.
%
For locations with the $\mathsf{MayW}$ tag, we cannot disregard $\tau_i$ and write $\mathsf{pc} \sqcup \iota$ because the index is only an overapproximation, i.e., we must avoid the risk of lowering the security level, as in runtime,  the contents of $S[i]$ may not be updated.
%
Moreover, the write footprint in the conclusion is $\mathsf{set}(o, \mathsf{wID}, \mathsf{mask})$, and the read footprint is $\delta_{\mathsf{read}}$.
%\farzaneh{please check the above. not sure about all the details.}
% This overapproximation is based on the assumption that any occurence of $\mathsf{tid}$ in $o$ is restricted to the warp under consideration $\mathsf{wID}$, and the set of (overapproximated) running threads, denoted by $\mathsf{mask}$.
% % 

We propose the following rule for handling $\mathsf{if}$ clauses in our type system. 
{\small\begin{mathpar}
    \infer*[Right=]{
        \mathsf{wID}; \mathsf{mask};\Sigma \vdash  e: \iota; \delta^0_{\mathsf{read}}\\
        \mathsf{wID}; \mathsf{mask} \wedge e;\Sigma, \pc \sqcup \iota \vdash  s_1 \dashv \Sigma_1;\delta^1_{\mathsf{write}}; \delta^1_{\mathsf{read}}\\
        \mathsf{wID}; \mathsf{mask} \wedge \neg e;\Sigma_1, \pc \sqcup \iota \vdash  s_2 \dashv \Sigma';\delta^2_{\mathsf{write}}; \delta^2_{\mathsf{read}}
        % \Sigma'= \Sigma_1 \sqcup \Sigma_2
     }{\mathsf{wID}; \mathsf{mask};\Sigma, \pc \vdash \mathsf{if}\, e \,\mathsf{then} \,s_1\, \mathsf{else}\, s_2 \dashv \Sigma';\delta^1_{\mathsf{write}} \cup \delta^2_{\mathsf{write}}; \delta^0_{\mathsf{read}} \cup \delta^1_{\mathsf{read}} \cup \delta^2_{\mathsf{read}}}
    \end{mathpar}
    }
    The rule types the conditional expression $e$ and both branches $s_1$ and $s_2$. Given that $e$ is of security type $\iota$, we update the program counter to $\mathsf{pc} \sqcup \iota$, which reflects the security implications of the condition,  when typing $s_1$ and $s_2$. 
    %
 Moreover, based on the condition $e$, some threads will execute $s_1$ while others will execute $s_2$. 
    % 
 To handle this, we update the thread masks for $s_1$ and $s_2$ judgments to $\mathsf{mask} \wedge e$ and $\mathsf{mask} \wedge \neg e$, respectively.
    %
 For example,  $\mathsf{mask} \wedge e$ means that threads satisfying both the condition in the current mask and the condition in $e$ will execute $s_1$. 
 Note that we do not have access to the runtime value of $e$ and thus can only have an overapproximation of which threads will satisfy $e$ at runtime. This is handled via the ``may writes'' tags, i.e., the access is only a ``may write'' if the mask is overapproximated.
%
When some threads run $s_1$ and others run $s_2$, the core will sequentialize them, meaning that we need to use the post-context for $s_1$ to serve as the pre-context for $s_2$, and the post-context for $s_2$ will be the final post-context of the entire if-clause.
%
The calculation of footprints for the if-clause is straightforward: it is a union of the footprints from both branches and the condition.
  
%   If one of the masks is false, its footprints will be empty and the postcontext will be equivelent to pre-context.
% where $\mathsf{set}(o)$ is a set consisting of all threads that can be identified by the expression $o$. For example, $\mathsf{set}(\mathsf{tid})$ refers to the set consisting of all threads, and $\mathsf{set}(c')$ is a singleton set consisting of thread with identifier $c'$.

% \begin{mathpar}
%     \infer*[Right=]{ \Sigma \vdash e: \iota\\
%     % x= \bigsqcup_{i \in \mathsf{wID}\cup \mathsf{mask}}x_i
%      }{\mathsf{wID}; \mathsf{mask};\Sigma; \pc \vdash S[\mathsf{tid}]\leftarrow e \dashv \Sigma[S[i]\mapsto \pc \sqcup \iota]_{i \in \mathsf{wID} \cup \mathsf{mask}}}
%     \end{mathpar}

% More generally, we can write

% \begin{mathpar}
%     \infer*[Right=]{ \Sigma \vdash e: \iota\\
%     % x= \bigsqcup_{i \in \mathsf{wID}\cup \mathsf{mask}}x_i
%      }{\mathsf{wID}; \mathsf{mask};\Sigma; \pc \vdash S[o]\leftarrow e \dashv \Sigma[S[i]\mapsto \pc \sqcup \iota]_{i \in \mathsf{wID} \cup \mathsf{mask} \cup \mathsf{set}(o)}}
%     \end{mathpar}


% \farzaneh{the above rule is not quite true, there should be an intersection involved. or define it as $\mathsf{set}(o,\mathsf{mask}, \mathsf{wID})$}


% For the reads from shared memory, we have:

% \begin{mathpar}
%     \infer*[Right=]{ 
%    \Sigma \vdash S[i]:i_\iota \\
%     \iota= \bigsqcup_{i \in \mathsf{wID}\cup \mathsf{mask} \cup \mathsf{set}(e)}\iota_i
%      }{\mathsf{wID}; \mathsf{mask};\Sigma; \pc \vdash X \leftarrow S[e] \dashv \Sigma[X \mapsto \iota \vee \pc]}
% \end{mathpar}
% Where $X$ is not a shared memory.
% When both reading and writing is from shared memory, we can write a similar rule by combining the two above rules.
% \begin{mathpar}
%     \infer*[Right=]{ \Sigma \vdash e: \iota\\
%     x= \bigsqcup_{i \in \mathsf{wID}\cup \mathsf{mask}}x_i
%      }{\mathsf{wID}; \mathsf{mask};\Sigma; \pc \vdash S[\mathsf{tid}]\leftarrow e \dashv \Sigma, S[1]:x \vee \pc, \cdots, S[n]:x \vee \pc}
%     \end{mathpar}

% To address the second challenge, regarding interleaving of warps, we need to combine the static typing judgments  for each individual warp into an analysis that accounts for how warps interact during execution.
% %
% To do this, we first break the code into several segments,  separated by synchronization points $\mathit{sync-threads}$.
% %
% Each segment represents a portion of code where threads within different warps can be run in any order of interleaving.
% %
% No warp can proceed further until all warps have completed their execution of the current segment. 
% %
% The core does fast switching between warps when warp has to wait for data for example.
% %
% This can be a source for data race, which results in an undefined behavior.
% %
% This can comlicate the dynamic taint tracking analysis when there are two accesses to a single location and at least one of them is a write.
% %
% One reason is for the simple fact that in the presence of data race all possible cimbinations are available.
% %
% The other reason which complicates it even further is that CUDA has weak memory model, meaning that we cannot rely on the sequential consistency in these cases either.
% %
% As such, in this task, we propose collecting an over-approximation of footprints for each warp and ensuring that they do not result in an underfined behavior.
% %
% We generalize the judgment as 
% $\mathsf{wID}; \mathsf{mask};\Sigma; \pc \vdash S  \dashv \Sigma; \delta_\mathsf{write}; \delta_\mathsf{read}$.
% For each segment, we make sure that the write footprints of each warp does not overlap with the read and write footprint of the others; two read footprints can overlap.
% %
% The red annotations in the above rules show show we collect such footprints.

To integrate taint tracking across all warps, we introduce the following rule that builds upon the judgment for each individual warp $\mathsf{wID}_k$ and each code segment $s_j$.

{\small
\begin{mathpar}
    \infer*[Right=]{
        \forall  \mathsf{wID}_k\in  \mathsf{warps}.\,\,
    \mathsf{wID}_k; \mathsf{true};\Sigma, \mathsf{Low}\vdash s_j \dashv \Sigma_{k};\delta_{\mathsf{write}_k}  ;\delta_{\mathsf{read}_k}\\
    \forall k,k'. \delta_{\mathsf{write}_k} \cap (\delta_{\mathsf{write}_{k'}} \cup \delta_{\mathsf{read}_{k'}})= \emptyset\\
    \Sigma'=\bigsqcup _{\mathsf{wID}_k\in  \mathsf{warps}}\Sigma_k
    }
     {\Sigma \vdash s_j \dashv \Sigma'}
    \end{mathpar}
}

Since segment separations via \texttt{syncthreads} occur outside of if-clauses, we know that the program counter ($\mathsf{pc}$) is initially low for all warps, and all threads are active at the start.
%
The first premise of the rule constructs a typing derivation for every warp, calculating their memory footprint and post-context; in the premise $\mathsf{warps}$ refers to the set of all warps, which we assume is predetermined and static (or at least that we have an upper bound on the number of warps that will be allocated).
% 
The rule builds a typing derivation for all warps in the set of all warps, calculating their footprint and also their post-context. We assume that the set of all warps is preset statically.
%
To ensure there are no data races, the second premise checks that the footprints of different warps do not overlap in a way that could result in undefined behavior, particularly when at least one warp performs a write operation on a particular location.
%
 The post-context for all warps is then computed in the third premise as the lattice join of the post-contexts for each individual warp.
%\farzaneh{are warps known statically?}


% We then establish taint tracking for each individual warp $\mathsf{wID}$ and each segment of the code $s_j$, by building the derivation for the judgment
% $\mathsf{wID};\mathsf{true};  \Sigma; \mathsf{Low}\vdash s_j \dashv \Sigma'$, starting from the mask $\mathsf{true}$ (indicating that all threads in the warp are active) and an initial program counter $\mathsf{Low}$. 
%

% To combine the derivations, we model the typing judgment $\mathsf{wID};\mathsf{true};  \Sigma; \mathsf{Low}\vdash s_j \dashv \Sigma'$ as a function $\mathsf{static-type} (\mathsf{wID}, \Sigma, s)$ from the pre-context $\Sigma$, the warp id $\mathsf{wID}$, and the $j$-th segment of the code to the post context $\Sigma'$.
%     %
% We build the pre-context $\Sigma$ by attaching a security type variable to all memory locations, making the derivations polymorphic in the security level of their pre-context.
    % %
    % $\Psi; \mathsf{wID}; \mathsf{true};  \Sigma; \mathsf{Low}\vdash s_i \dashv \Sigma'$.
    % %
    % The function $\mathsf{static-type} (\mathsf{wID}, \Sigma, s_i)$ returns the context $\Sigma'$.
    % %
    % For each segment $s_j$, we consider all possible permutations of warps, and calculate the result of an iterative application of function $\mathsf{static-type}$ based on each permutation. 
    % %
    % The first input ($\mathsf{wID}$) for the $k$-th iteration of the function is the warp id of the $k$-th warp in the permutation.
    % %
    % Moreover, the post-context of the $k$-th iteration (for the $k$-th warp in the permutation) serves as the pre-context of the $k+1$-th iteration.
    % %
    % This chaining ensures that the taint propagation correctly reflects the order of execution of the warps, as each warp's execution can potentially affect the subsequent warp's execution in a way that modifies the shared memory's security levels.
    % % 
    % By applying $\mathsf{static-type}$ iteratively across each permutation of warps, we track the effect of all possible interleavings.

    % We calculate the post-context of each permutation and each segment using a dynamic algorithm implementation.
    % %
    % The final post-context for each segment will be the join of the post-context of all permutations (the worst case scenario).
    %
    % The final post-context for each segment will be the join of the post-context of all warps.
    % %
    Once the post-contexts for all segments have been computed, we combine them to determine the overall final context for the entire program. 
    %
   The rule below illustrates how the post-context of the previous segment is used as the starting post-context for the next segment.
    {\small
    \begin{mathpar}
        \infer*[Right=]{
          \Sigma_0 \vdash s_1 \dashv \Sigma_1\\
          \Sigma_1 \vdash s_2 \dashv \Sigma_2\\
          \cdots\\
          \Sigma_{n-1} \vdash s_n \dashv \Sigma_n\\
        }
         {\Sigma_0 \vdash s_1; \texttt{syncthreads}; s_2; \cdots; \texttt{syncthreads}; s_n \dashv \Sigma_n}
        \end{mathpar}
    }
    %
    The goal of this rule is to calculate a final context that overapproximates the security state of the program when kernel execution terminates. 
    %
    If the final context $\Sigma_n$ has low-security levels across all shared memory locations, we can conclude that the EoK attack is not possible.
%
Similarly, we can establish results for EoC attacks by combining different code blocks.
%
% Similarly, we build the system for global memory at the end of the context.

To prove the soundness of our type system, we aim to demonstrate a noninterference property. This would show that running the program with different secrets, but the same public information, results in the same final state in shared memory. 
Formally, we want to prove that, given two low-equivalent initial memory states (where the public information is identical and the secrets differ), if the first run of the program generates a final state, then any possible execution of the second run will produce an identical final state.






\subsubsection{Task 2.b} 
The static algorithm above can be too restrictive since it does not have access to runtime values and also has to consider all possible interleavings of warps, while in runtime, depending on the scheduler, only some of these interleavings may actually happen. 
%
In this task, we plan to build a flow-sensitive hybrid runtime monitor inspired by Russo and Sabelfeld~\cite{russo2010CSF}
that also accepts all statically typed programs in Task 2a.
%

We plan to model the dynamic monitor using a set of operational semantics rules.
%
The rules work on configurations of the form $\langle (s_1, \cdots, s_n); s, \sigma \rangle  \mid \langle \Sigma, (\Sigma_1, \cdots, \Sigma_n)\rangle$, where $s_k$ represents the running code on the warp $\mathsf{wID}_k$ for the current seqment, $s$ is the running code representing a sequence of segments being executed after the current segment is complete, $\sigma$ is the runtime memory state, which assigns values to locations,  $\Sigma $ is the state of the monitor, which  assigns security levels to locations, 
and $\Sigma_k$ is a dynamic version of the pc level that tracks the nested conditional branches for each warp $\mathsf{wID}_k$.
% $R$ is the queue of warps ordered by the scheduler that are currently waiting to run the current segment, and $C$ is the set of warps that have already completed running the current segment.
%
% \farzaneh{what is $\mu$? events of the monitor.}

The monitor's role is to track the propagation of security levels, updating the state as the program executes.
%
% However, for each segment and each warp, instead of tracking such propagation dynamically, we use the static typing function $\mathsf{static-type}$.
% %
% If the scheduler executes $\mathsf{wID}$ to run segment $s_j$, we update the state of the monitor via $\Sigma'=\mathsf{static-type}(\mathsf{wID}, \Sigma, s_j)$, where $\Sigma$ is the pre-state and $\Sigma'$ is the post-state.
% %
The following rule models this idea.
% The monitor state  $\Sigma$ is updated as warps modify shared memory locations.  For example, if a warp writes to a shared memory location, the monitor may increase the security level of that location depending on the security level of the contents (e.g., if it writes high-security data, the memory location's security level will be updated to high).


{\small\begin{mathpar}
    \infer*[Right=]{
       \langle s_k, \sigma \rangle \rightarrow_\alpha \langle s'_k, \sigma' \rangle  \\
       \alpha \vdash \langle \Sigma, \Sigma_k \rangle \rightarrow_\delta \langle \Sigma', \Sigma'_k \rangle
     }{\langle (s_1, \cdots s_k \cdots, s_n); s, \sigma \rangle  \mid \langle \Sigma, (\Sigma_1,\cdots \Sigma_k \cdots, \Sigma_n)\rangle\,\,\rightarrow_\delta \,\,\langle (s_1, \cdots s'_k \cdots, s_n); s, \sigma' \rangle  \mid \langle \Sigma', (\Sigma_1, \cdots \Sigma'_k \cdots, \Sigma_n)\rangle}
\end{mathpar}
}
The rule chooses to execute an instruction from warp $\mathsf{wID}_k$. 
%
The first premise is straightforward and uses a small-step semantics on a pair of code $s_k$ and memory state $\sigma$.
%
The premise also includes $\alpha$, a summary of actions taken by this step,
which will be used to update the state of the monitor in the second premise.
%
% $\delta$ in the second premise is the action of the monitor.
 %
For example if the next immediate step of $s_k$ is to write expression $e$ on location $S[10]$ in shared memory, the action $\alpha$ would be $w(e, S[10])$. In the second premise, the state of the monitor is updated by computing the read footprint of $e$. Based on the security level of the locations in this footprint and the current state of $\Sigma'_k$, the monitor adjusts the security level of $S[10]$ in $\Sigma$. 
%
If the next immediate step of $s_k$ is to execute an if-clause and take the first branch $s_1$, for example, the action $\alpha$ would be $b(e, s_2)$, where $e$ is the condition of the branch and $s_2$ is the second branch.
%
The monitor uses the statically calculated read footprint of $e$ and the write footprint of $s_2$, and updates the state of $\Sigma'_k$ accordingly.
%
Specifically, if $e$ accesses any high-secrecy location, the monitor temporarily retains the locations in the write footprint of $s_2$ in the stack until the end of the branch. Once the branch completes, the security level of these locations is updated to high.
%
This approach helps prevent indirect leaks through non-executed branches, as discussed in \cite{russo2010CSF}.

% More interestingly, if the next immediate step of $s_k$ is to execute an if-clause taking the first branch $s_1$, for example, the  

When all warps complete a particular segment, we continue with the rest of the next immediate segment $s'$ without changing the state of the monitor. In the following rule, $\mathsf{stop}$ and $\overline{\emptyset}$ state that all warps completed the code of the segment, and none of them is inside a branch, respectively. 
{\small \begin{mathpar}
    \infer*[Right=]{
      }{\langle \overline{\mathsf{stop}};(s';\texttt{syncthreads};\, s), \sigma \rangle  \mid  \langle \Sigma, \overline{\emptyset}\rangle\,\, \rightarrow\,\, \langle (s', \cdots , s'); s, \sigma \rangle  \mid  \langle \Sigma, \overline{\emptyset} \rangle}
\end{mathpar}
}
Finally, the rule below ensures that the kernel terminates only when every shared memory location is non-tainted; the monitor verifies that all shared memory locations are annotated with a security level of $\mathsf{Low}$.

{\small\begin{mathpar}
  \infer*[Right=]{
    \forall S[i]:\iota \in \Sigma.\, \iota \sqsubseteq \mathsf{Low} 
    }{\langle \overline{\mathsf{stop}};\cdot, \sigma \rangle  \mid  \langle \Sigma, \overline{\emptyset}\rangle\,\, \mathsf{success}}
\end{mathpar}
}

If the condition of having low-security levels across all shared memory locations does not hold, we can design the monitor to either raise an alarm or take corrective action by zeroizing the tainted memory locations. 
%
Alternatively, the monitor could prevent future access to such tainted memory locations by restricting their access to any kernel outside the application, effectively isolating the sensitive data from potential attackers.
%
Similarly, we will design rules to mitigate the EoC attack concerning the global memory.


We will prove that our dynamic monitor is sound, i.e., given two low-equivalent initial memory states, if the first run of the program generates a final state, then there exists a possible execution of the second run that produces an identical final state. 
%
This result is weaker than the result we aim to prove for the static analysis, as it relies on an existential quantifier for the second run, whereas the static analysis uses a universal quantifier, considering all possible interleavings of threads or warps.
%
We also formally prove that all programs that are well-typed in the static analysis are dynamically secure.

% 
% Raising an alarm would notify the system or the user that a security violation may occurr.
% % 
% In the latter approach, 
% Alternatively, we can design the monitor to take action to zeroize the tainted memory locations.
%
% If not done carefully, the latter approach can be a source of implicit flows. 
% %
% In particular, if the monitor simply writes a random constant on the tainted location which is not possible for any given secret, the attacker knowing the pre-state can infer a secret based on which location is zeroized.
%
% still observe the difference between the value written by the monitor and what she expected if the secret was something else and the memory locations end up not needing to be zeroized.

% Same ideas hold for the global memory.
% It will be interesting to mix shared and global memory.
% \farzaneh{do we need more discussion on this?}

% \farzaneh{We can also propose a hybrid dyamic-static monitor for the warp-level. Should we? See~\cite{russo2010CSF}}


\subsubsection{Task 2.c.} 
We plan to implement a Taint Analysis Framework that integrates the ideas developed in Tasks 2a and 2b. 
%
This framework will be implemented in OCaml.
%
The framework will support both static taint tracking which prevents undefined behavior (from Task 2a) and dynamic monitoring (from Task 2b).
%
We plan to use the implemented framework to evaluate the performance, e.g., execution time and resource utilization, of our proposed approaches. 
% 
We will use the AES encryption algorithm as a benchmark.
%
We anticipate that using a static footprint analysis in the dynamic monitor will enable better performance compared to prior taint tracking systems, as it will help reduce unnecessary runtime calculations.
\farzaneh{Do we need to add more?}

