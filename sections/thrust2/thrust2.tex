\subsection{Motivating Example}
\farzaneh{move to the intro if needed: GPU memory is split into several regions, onchip and off-chip. On-chip memory consists of registers,
caches, and shared memory.
Off-chip memory
is GDDR SGRAM, which is logically distributed into
texture memory, constant memory, local memory, and
global memory.
Both texture memory and constant memory are read-only during the GPU kernel execution. Local memory is private to each thread and thus does not play a role in any of the two attack models. Therefore, in this proposal we focus on registers, shared memory, and global memory.
}
\farzaneh{Merge the paragraph here or add to the first thrust:
The victim sends its plaintext to the server through a secure channel that the attacker cannot observe.
% 
The server, then, uses a private cipher key to encrypt the plaintext into a ciphertext that is public and, therefore, visible to the attacker.
%
The attacker, using the same server for a different purpose, aims to get access to either the victim's plaintext or the cipher key used by the server;  accessing either of these would allow the attacker to decrypt the plaintext, as it can observe the public ciphertext.
% 
Even accessing a few words of plaintext or the cipher key is regarded as a security leak, as it allows the attacker to execute a faster brute-force attack on the remaining data.
}
\farzaneh{Amdahl's law + GPUs are expensive -> one way is to share, e.g., MPS}

%
The Multi-Process Service (MPS) is a software solution designed to efficiently manage multiple processes sharing a GPU allowing to run them concurrently~\cite{anasic2014CAN, NVDIA2013, li2011ICPP}.
%
% It is designed to enhance performance when a single application process underutilizes the GPU's compute capacity.
%
Processes sharing the GPU send their requests to the MPS, where these requests are queued and processed by the GPU using a FIFO scheduling policy.
% 
MPS is designed to enhance performance when a single application process underutilizes the GPU's compute capacity and thus the GPU is shared between multiple processes.
%
However, one of its key limitations is that memory allocated by any of the client processes is accessible to all other client processes, which compromises memory isolation between them.
%
% New requests can be accepted even while another application is currently executing a kernel on the GPU.


In this thrust, we consider the scenario in which both the victim and attacker are clients of a shared GPU server (see Figure~\ref{fig:th2-attack}). 
%
If the attacker's application runs concurrently with the victim's, it can potentially access private information that is left on the GPU memory by the victim's application.
%
This potential leak occurs if the memory locations allocated to the attacker overlap with those used by the victim.
%
To achieve this, the attacker only needs to access the GPU memory through regular allocation, deallocation, and read/write operations without requiring any special privileges.
%
The details of the attack vary depending on whether the victim's information is located in the shared or global memory.
%
We describe the details using an example similar to Thrust 1, where the victim uses the GPU server to encrypt sensitive plaintext.

\paragraph{Shared memory.}
% Shared memory is available to the programmer, often treated as a software cache.
%
In this scenario, both the victim and the adversary client processes submit their requests to the MPS. 
% 
Let us assume that the scheduler executes the attacker's request immediately after the victim's, granting the attacker exclusive access to the GPU and its memory.
% 
In such a situation, the attacker could gain access to any data left by the victim in the shared memory. 
% 
For instance, if the victim stores an encryption key in the shared memory, the adversary can retrieve it.
% 
The key is that such interleaving may occur before the process exits, as determined by the scheduler. 
% 
As such, even if the programmer attempts to clear the memory before exiting, this interleaved access can still result in information leakage. 




\paragraph{Global memory.}

% 
The main point is that in both form of these attacks, data persists after deallocation. CudaFree is not enough.
For what concerns registers, a feature that could taint memory isolation is that registers can be used to access global memory as well: in fact, a GPU feature (named register
spilling [Micikevicius 2011]) allows one to map a large number of kernel variables onto
a small number of registers. When the GPU runs out of hardware registers, it can
transparently leverage global memory instead.


% Local
% memory is thread-private, and is most commonly used
% for register spilling. 
% 
With the attack in~\cite{}, we are able to leak only the final state of the previous GPU process. 
%
It is due to the exclusive access granted by the driver to host threads that access the GPU; only one cudaContext is
allowed to access the GPU at a given time.
%
Sometimes the final state is of high confidentiallity: the decryption of a
ciphertext, the output of a risk-analysis function.
%
But sometimes the final state is of low confidentiality of public:
encryption algorithm.
%
We can capture this by our taint analysis.
%
This is not only the "output". 
%
But the residues in the memory in the final state (side effects of the program).
%
For example, if in the final state of an encryption algorithm, we still have the plaintext on the global memory, the attacker can get access to it.







Global memory can be set and cleared through API
functions, with overhead similar to that of running a GPU
kernel, but local memory, shared memory, and registers
are only accessible from within a kernel function, and allocated and deallocated by the driver. These three memory types can only be reliably cleared through instrumentation. Moreover, local memory and registers are
managed by compilers and they can only be cleared by
compile-time instrumentation.
Sensitive information can also propagate to different data storage locations on GPU: memory, software
caches, and registers. An example is the advanced encryption standard (AES), in which the key and the plain
text to be encrypted may reside in different types of
memory [25]. They can be stored in global memory as
allocated data objects and in registers as program execution operands.
Service (MPS), one application can peek into the memory of another application, documented in NVIDIA's
MPS manual at Section 2.3.3.1, “An out-of-range read
in a CUDA Kernel can access CUDA-accessible memory modified by another process, and will not trigger
an error, leading to undefined behavior.” When two applications do not run simultaneously, in which case every application will get a serially scheduled time-slice
on the whole GPU, information leaking is still possible. The second running application can read data left
by the first running application if its allocated memory
locations happen to overlap with those of the first one.
This vulnerability has been detailed in several recent
works. 



% \farzaneh{rewrite and move to intro?}
% The kernel programmer usually aims at writing the fastest possible code without devoting time to address security/isolation issues that might hamper performance.
% In fact, the CUDA Runtime copies the data to the GPU's global memory on behalf of the programmer when it
% is more convenient.
% This is important since it shows that even code implemented by “experts” actually shows the
% same deficiencies as regards security.
%  This finding, together with the others reported in the paper, call for solutions
% to this severe vulnerability.


\subsection{Proposed work}

The taintable sources are program inputs
given by the users and reside in the global memory
on GPUs. Examples include face recognition photos, a plain-text message, and encryption key.

It enables data protection that clears sensitive (tainted) data objects at the end of their life range as well as detects leak of the sensitive data in the midst of program execution.


Dynamic taint tracking is expensive, particularly if we don't want to consider a single bit and do genreic taint tracking.
% 
In prior work they considered making it better in performance by considering things specific to GPU.
%
We want to do so by combining static analysis and dynamic analysis.
%
For the static analysis we can capture the flow of information inside each application.
%
Since the applications are scheduled nondeterministically though, and can be used by different clients, we cannot use static taint tracking for interleaving applications in an efficient way. Otherwsise it will be too restrictive.
%
We built upon the static information flow control built prior.
This happens when both kernels are accessing the shared memory, and one is safe while the other one is not. written in shared, is already given by the first thrust.
\subsubsection{Task 2.a.} 
For the global memory, even after they are done, mallocfree will not save it.  When allocated statically it is not clear what is going to allocated and it is  dynamic choice.
This is a place where a dynamic tainting comes in.
one way to do it is to keep the security high as is even after mallocfree, but this will be restrictive. as they remain tainted.

Assuring security in the presence of context switching between programs of GPU
Taint tracking analysis for the source language and modeling a dynamic monitor at the hardware level.

We first describe how dynamic taint analysis
is implemented by monitoring the execution of a program.

Taint policy actions, whether it be taint propagation, introduction,
or checking, are added to the operational semantics rules.

We redefine values in our language to be tuples of the form $\langle v, \tau \rangle$ where v is a value in the initial language, and $\tau$ is the taint status of $v$. 
$\Delta_\tau$ keeps track of the taint status of
scalar variables. $\mu_\tau$ keeps track of the taint status of memory cells. 
At the beginning they are initialized so that all values are marked
untainted. keep the taint status for all
variables and memory cells, and are used to derive the taint
status for all values during execution.

A taint policy specifies three properties: how new taint is
introduced to a program, how taint propagates as instructions
execute, and how taint is checked during execution
A taint policy will also typically
distinguish between different input sources.
For example, an
internet-facing network input source may always introduce
taint, while a file descriptor that reads from a trusted
configuration file may not.
\farzaneh{We can ask the cloud client to provide the level of security of their data and tell us which parts they want to keep secure.}
Taint propagation rules specify the taint
status for data derived from tainted or untainted operands.
Since taint is a bit, propositional logic is usually used to
express the propagation policy, e

Taint Checking. Taint status values are often used to
determine the runtime behavior of a program, e.g., an attack
detector may halt execution if a jump target address is
tainted. , we perform checking by adding the
policy to the premise of the operational semantics


Any program value whose computation depends on data derived from a taint source is considered tainted. A taint policy P determines exactly how taint flows as a program executes, what sorts of operations introduce new taint, and
what checks are performed on tainted values.

Two types of errors can occur in dynamic taint analysis.
First, dynamic taint analysis can mark a value as tainted
when it is not derived from a taint source. We say that such
a value is overtainted. For example, in an attack detection
application overtainting will typically result in reporting
an attack when no attack occurred. Second, dynamic taint
analysis can miss the information flow from a source to a
sink, which we call undertainting. In the attack detection
scenario, undertainting means the system missed a real
attack. A dynamic taint analysis system is precise if no
undertainting or overtainting occurs.
\subsubsection{Task 2.b} 
The static algorithm above can be too restrictive since it has to consider all possible interleavings of warps, while in runtime, depending on the scheduler, only some of these interleavings may actually happen. 
%
(\farzaneh{is this true?})
In this task, we plan to build a flow-sensitive dynamic runtime monitor
that works on top of the static type system above.
%
This means that we still use the function $\mathsf{static-type}$ as described in the previous task for each segment of the code and  each warp, but use a dynamic monitor to track the actual interleavings of warps and find the real permutation of the functions for each segment.

We plan to model the dynamic monitor using a set of operational semantics rules.
%
The rules work on the configuration of the form $\langle s, \sigma \rangle \mid \langle R,C \rangle  \mid_\mu  \Sigma$, where $s$ is the running code representing a sequence of segments being executed in order, $\sigma$ is the runtime memory state, which assigns values to locations,  $\Sigma $ is the state of the monitor, which  assigns security levels to locations, $R$ is the queue of warps ordered by the scheduler that are currently waiting to run the current segment, and $C$ is the set of warps that have already completed running the current segment.
%
\farzaneh{what is $\mu$? events of the monitor.}

The monitor's role is to track the propagation of security levels, updating the state as the program executes.
%
However, for each segment and each warp, instead of tracking such propagation dynamically, we use the static typing function $\mathsf{static-type}$.
%
If the scheduler executes $\mathsf{wID}$ to run segment $s_j$, we update the state of the monitor via $\Sigma'=\mathsf{static-type}(\mathsf{wID}, \Sigma, s_j)$, where $\Sigma$ is the pre-state and $\Sigma'$ is the post-state.
% %
The following rule models this idea.
% The monitor state  $\Sigma$ is updated as warps modify shared memory locations.  For example, if a warp writes to a shared memory location, the monitor may increase the security level of that location depending on the security level of the contents (e.g., if it writes high-security data, the memory location's security level will be updated to high).


\begin{mathpar}
    \infer*[Right=]{
       R= \mathsf{wID};R'\\ 
       {\mathsf{wID}}   \vdash \langle s_j, \sigma \rangle \Downarrow\langle \mathsf{skip}, \sigma' \rangle \\
       \Sigma'=\mathsf{static-type}(\mathsf{wID}, \Sigma, s_j)
     }{\langle s_j;\bar{s}, \sigma \rangle \mid \langle R, C\rangle \mid_\mu  \Sigma\,\, \rightarrow\,\, \langle s_j;\bar{s}, \sigma' \rangle \mid \langle R', \mathsf{wID};C\rangle \mid_\mu  \Sigma'\,}
\end{mathpar}

When all warps complete a particular segment, we continue with the rest of the segments, without changing the state of the monitor.
%
The queue of the warps for the next segment is determined by the scheduler.
\begin{mathpar}
    \infer*[Right=]{
        R= \mathit{Schedule}(C)
      }{\langle s;\bar{s}, \sigma \rangle \mid \langle \epsilon, C\rangle \mid_\mu  \Sigma\,\, \rightarrow\,\, \langle \bar{s}, \sigma \rangle \mid \langle R;\epsilon \rangle \mid_\mu  \Sigma\,}
\end{mathpar}

The rule ensures that the kernel terminates only when every shared memory location is non-tainted; the monitor verifies that all shared memory locations are annotated with a security level of $\mathsf{Low}$.

\begin{mathpar}
    \infer*[Right=]{
     \forall S[i]:\iota \in \Sigma.\, \iota \sqsubseteq \mathsf{Low}   
      }{\langle \mathsf{skip}, \sigma \rangle \mid \langle R, C\rangle \mid_\mu  \Sigma\,\, \rightarrow\,\, \langle \mathsf{stop}, \sigma \rangle \mid \langle R;C \rangle \mid_{\mu;\mathsf{secure}}  \Sigma\,}
\end{mathpar}

If the condition of having low-security levels across all shared memory locations does not hold, we can design the monitor to either raise an alarm or take corrective action by zeroizing the tainted memory locations. 
% 
% Raising an alarm would notify the system or the user that a security violation may occurr.
% % 
% In the latter approach, 
% Alternatively, we can design the monitor to take action to zeroize the tainted memory locations.
%
If not done carefully, the latter approach can be a source of implicit flows. 
%
In particular, if the monitor simply writes a random constant on the tainted location, the attacker can infer a secret based on which a location might be zeroized or not.


% still observe the difference between the value written by the monitor and what she expected if the secret was something else and the memory locations end up not needing to be zeroized.

Same ideas hold for the global memory.
It will be interesting to mix shared and global memory.
\farzaneh{do we need more discussion on this?}

\farzaneh{We can also propose a hybrid dyamic-static monitor for the warp-level. Should we? See~\cite{russo2010CSF}}


\subsubsection{Task 2.c.} 
We plan to implement a Taint Analysis Framework that integrates the ideas developed in Tasks 2a and 2b. 
%
This framework will be implemented in OCaml.
%
The framework will support both static taint tracking which prevents undefined behavior (from Task 2a) and dynamic monitoring (from Task 2b).
%
We plan to use the implemented framework to evaluate the performance, e.g., execution time and resource utilization, of our proposed approaches. 
% 
We will use the AES encryption algorithm as a benchmark.
%
We anticipate that using a static footprint analysis in the dynamic monitor will enable better performance compared to prior taint tracking systems, as it will help reduce unnecessary runtime calculations.
\farzaneh{Do we need to add more?}

