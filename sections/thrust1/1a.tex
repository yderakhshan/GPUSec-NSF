\subsubsection{Task 1.a. Information Flow Control Type System for GPU Timing Attacks}

Our goal in this task is to statically ensure information security of GPU kernels, and in particular their invulnerability to timing attacks.
%
Our security condition in this task will be {\em cost-sensitive noninterference} \stefan{Is there an existing term for this?} for individual GPU kernels.
%
We define this condition as follows: assign a {\em security level} to every local variable and array in a kernel indicating the secrecy of its contents---for example purposes, we will consider security levels High (high secrecy) and $\bot$ or Low (not secret), but the techniques apply to general lattices of security levels.
%
Consider two executions of a program, starting with memories that agree on variables of security level less than~$\seclevel$ but may differ on higher-security variables.
%
We call such starting points {\em low-equivalent} states.
%
Cost-sensitive noninterference at a security level~$\seclevel$ requires that two executions of a program, starting with low-equivalent states, are indistinguishable in their input-output behavior, effect on low-security memory locations, and observable cost behavior.
%
For most of the examples in this proposal, ``observable cost behavior'' means number of bank conflicts, but in our proposed work, we plan for this to also include number of memory transactions and other performance characteristics closely tied to program data.

In this task, we ensure cost-sensitive noninterference through two main mechanisms.
%
First, we develop a resource-aware Relational Hoare Logic for quantitatively bounding the difference in execution cost between two evaluation of a statement or expression (e.g., that differ only on high-security memory locatins).
%
This logic is sufficient to ensure noninterference and is quite powerful in specifying fine-grained policies.
%
However, it suffers from the same drawbacks as program logics in general: they are often not decidable and require significant human interaction and/or advanced heuristics to prove complex properties such as noninterference.
%
To complement these drawbacks, we will develop an information flow control (IFC) type system that tracks security levels of expressions and statements throughout the program and disallows operations that might violate cost-sensitive noninterference.
%
At key points, for example at a conditional where the condition depends on high-security information, the type system must ensure that the execution costs (e.g., bank conflicts, memory accesses) of the two branches are identical.
%
It does this by ``calling out'' to the Relational Hoare Logic.
%
Uses of the more general logic are therefore localized, reducing the burden on programmers and analysis tools.

We show the soundness of our approach in two steps.
%
First, we show that the Relational Hoare Logic is sound with respect to an operational model of GPU execution.
%
As the model, we plan to use the operational semantics of MiniCUDA~\cite{MullerHo21}, which also track abstract execution cost including bank conflicts and memory accesses.
%
Second, we will show the type system sound with respect to the Relational Hoare Logic, i.e., a well-typed program is related to itself in that any two executions with low-equivalent starting states have the same cost and observable behavior.



\paragraph{Resource aware - Relational Hoare Logic.}
The judgment for our resource-aware Relational Hoare Logic will be of the form 
%
\[\{\Phi; Q; X\} p_1 \sim p_2 \{\Phi'; Q'; X'\}\]
%
where $p_1$ and $p_2$ are programs the we want to relate (they may be the same program, in which case we relate two executions).%
The logical preconditon $\Phi$  describes the relation between the two states before running programs $p_1$ and $p_2$. 
%
The {\em potential function} $Q$ maps program states to rational numbers.
%
Such a function is familiar from the potential or ``physicist's'' method of amortized analysis, where certain operations accrue potential (analogous to potential energy in physics) in the program state that can then be used to amortize the cost of periodic expensive operations.
%
Our Relational Hoare Logic utilizes a non-standard interpretation of the potential function: the potential actually pays for the {\em difference} between the two execution costs.
%
If the two programs evaluate in lockstep, performing identical memory transactions and other observable operations, no cost is accrued.
%
As discussed earlier \stefan{keep track of where}, we only want to reason about the values of variables that are known to be uniform across all threads; this means that only the states of such variables are included in~$Q$.
%
We keep track of these variables using the third component of the precondition, $X$, which is itself a pair $(X_1, X_2)$ of sets of variables.
%
Sets~$X_1$ and~$X_2$ contain the variables known to be uniform before executing~$p_1$ and~$p_2$, respectively.
%
Moving to the postcondition, $\Phi'$ describes the relation between the two states after running programs $p_1$ and $p_2$; $Q'$ is a potential function for the remaining potential after paying for any differences in cost between $p_1$ and $p_2$, and $X'$ contains the sets of uniform variables in the two executions after executing the two statements.

We will need two forms of rules, {\em synchronous} and {\em unscynchronous}.
%
The synchronous rules relate two executions of the same form of statement;
for example, here is the rule that relates two assignments to shared memory. 
\[
\infer[]{ \Phi[e_1/S_1[o_1]][e_2/S_2[o_2]] \Rightarrow \mathtt{diff\_conflicts}(o_1,o_2) \le n\\  \Phi \vdash e_1 \sim e_2 : m}{\{\Phi; Q+n+m; X\} S_1[o_1] \leftarrow e_1 \sim S_2[o_2] \leftarrow e_2 \{\Phi; Q; X\}  }
\]
The premise~$\Phi \Rightarrow \mathtt{diff\_conflicts}(o_1,o_2) \le n$ indicates that, under the logical conditions in~$\Phi$, it is possible to reason that the difference in the number of bank conflicts between the two indices~$o_1$ and~$o_2$ is less than~$n$.
%
For example, if~$\Phi$ guarantees that~$o_1 = o_2$, then we can conclude~$ \mathtt{diff\_conflicts}(o_1,o_2) = 0$.
%
The auxiliary judgment~$\Phi \vdash e_1 \sim e_2 : m$ states that the difference in the number of bank conflicts triggered by~$e_1$ and~$e_2$ is bounded by~$m$.
%
The rules for this judgment mainly relate shared memory accesses in subexpressions of~$e_1$ and~$e_2$ using~$\mathit{diff\_conflicts}$.
%
In the conclusion of the rule, the main change between the pre- and post-conditions is a cost of~$n + m$ to pay for the maximum possible difference in bank conflicts between the two programs.
%
We also include the information about the assignment in~$\Phi$ by substituting the assigned expressions for the target locations in the precondition, as is standard in Hoare Logic.

The asynchronous rules relate two different forms of statements, and are necessary for, e.g., relating the two branches of a conditional to guarantee that the choice of branch can't leak information through timing channels.

We prove soundness of the Relational Hoare Logic by comparing it to the cost-annotated operational model of CUDA execution developed by Muller and Hoffmann~\cite{MullerHo21}.
%
The operational semantics judgment is~$\sigma; p \Downarrow^{\mathcal{T}}_m \sigma'; n$.
%
In this judgment,~$\sigma$ and~$\sigma'$ are {\em states}, i.e., the values of local variables and arrays, before and after execution.
%
We write~$\Phi \vdash (\sigma_1, \sigma_2)$ to mean that a pair of states satisfies a condition~$\Phi$ in that the states and the relation between them satisfy all of the logical conditions specified in~$\Phi$.
%
The program~$p$ executes with the set~$\mathcal{T}$ of active threads (this may be a subset of all threads if warps have diverged).
%
The judgment also indicates the cost~$n$ of execution in units specified by a {\em resource metric}~$m$, which assigns costs to program operations and conditions---in our example, we would use a resource metric that assigns a cost of~$m$ to an~$m$-way bank conflict and a cost of zero to all other operations.
%
The soundness theorem for the logic states that, if we execute~$p_1$ and~$p_2$ starting with states that satisfy~$\Phi$, the difference in potential predicted by the logic overestimates the actual cost difference between the two executions:

\begin{theorem}[Soundness of Relational Hoare Logic]\label{thm:logic-sound}
If~$\Phi \vdash (\sigma_1, \sigma_2)$ and
$\{\Phi, Q, \{\}\} p_1 \sim p_2 \{\Phi', Q', X'\}$
and $\sigma_1; p_1 \Downarrow^{\mathcal{T}}_m \sigma_1'; n_1$
and $\sigma_2; p_2 \Downarrow^{\mathcal{T}}_m \sigma_2'; n_2$
then $|n_1 - n_2| \leq Q(\sigma_1, \sigma_2) - Q'(\sigma'_1, \sigma_2')$.
\end{theorem}


\paragraph{IFC type system.}
The Relational Hoare Logic described above is sufficient to guarantee noninterferece.
%
We can guarantee that a program~$p$ satisfies noninterference by requiring the
provability of the triple
\[\{\Phi; 0; \{\}\} p \sim p \{\Phi'; 0; \{\}\}\]
where~$\Phi$ is a condition that requires only that the two initial states be low-equivalent.
%
The soundness theorem guarantees that if the triple is provable, then the two executions of~$p$ do not differ in their resource usage.
%
However, proving such a triple may be quite cumbersome and there is no guarantee that doing so would be possible without substantial human effort.
%
We therefore develop a more conservative, but simpler and decidable, type system.

Information flow control (IFC) type systems typically extend conventional type systems by tracking the security level of variables in addition to their data types.
%
In doing so, the type system can, for example, restrict assignments that would
cause high-security information to flow to a low-security variable.
%
In CUDA, we need to track not just the security level of variables and
expressions but also
whether or not they are uniform across threads, because conditioning on
non-uniform expressions can cause leaks, as shown in \stefan{Example}.
%
Types in our type system are~$\tau_{\lpair{\seclevel}{\divergence}}$
where~$\tau$ is a standard data type such as $\mathsf{int}$, and
$\seclevel$ is a security level (we will use~$\bot$ for low-security and~$\top$
for high security; in general these can be values from an arbitrary security
lattice).
%
The component~$\divergence$ is either~$\dunif$ meaning ``uniform across
threads'' or~$\ddiv$ meaning ``possibly divergent (non-uniform) across threads.''
%
We can also treat pairs~$\lpair{\seclevel}{\divergence}$ as elements~$P$ of a
single lattice formed by taking the Cartesian product of the security
lattice with $\{\ddiv, \dunif\}$ and extending the join and ordering operators
in the natural way (uniform values may flow to non-uniform variables but not
vice versa).

Our typing judgment for CUDA statements~$s$ is
$\Sigma, \pc \vdash s$,
where~$\Sigma$ is a context providing the types of constants and variables,
and $\pc$ is the security level of the program counter, which makes the type
system flow-sensitive by indicating whether the control flow of the program
has been influenced by high-security information.


As an example, a general rule for assignments to shared memory might be:
\[
\infer[]
{ \Sigma(S) = \mathsf{arr}(\tau)_{\lpair{\seclevel_S}{\divergence_S}}\\
\Sigma, \pc \vdash o : \mathsf{int}_{\lpair{\bot}{\divergence_o}}\\
\Sigma, \pc \vdash e : \tau_{\lpair{\seclevel_e}{\divergence_e}}\\
(\pc \ljoin \seclevel_e) \secle \seclevel_S
}
{
\Sigma, \pc \vdash S[o] \leftarrow e
}
\]
The assignment is permitted as long as 1) the array index~$o$ is public
(since using a secret index could leak information through the number of
bank conflicts)
and 2) the security level of array~$S$ is at least as high as the
join of~$\pc$ and the security level of~$e$.
%
Clearly~$\seclevel_S$ must be higher than~$\seclevel_e$ as we are writing~$e$
into~$S$.
%
Using the level of the program counter prevents so-called {\em indirect flows}
such as this one, which also writes the value of the boolean~$b$ into~$S[0]$:
\begin{lstlisting}[language=C]
if (b) S[0] = 1;
else S[0] = 0;
\end{lstlisting}

Note that writing to shared memory when the program counter is impacted by
secret information is allowed; disallowing this would be safe, but unnecessarily
restrictive.
%
Instead, we prevent indirect leaks through bank conflicts with the rules
for \texttt{if}.

When conditioning on public information, it suffices to check that both
branches are well-typed:
\[
\infer[]
{\Sigma, \pc \vdash e : \mathsf{bool}_{\lpair{\bot}{\divergence_e}}\\
\Sigma, \pc \vdash s_1\\
\Sigma, \pc \vdash s_2
}
{\Sigma, \pc \vdash \mathsf{if}~e~\mathsf{then}~s_1~\mathsf{else}~s_2}
\]
Note that both branches are typed under~$\pc$ because conditioning on
public information does not increase the secrecy level of the program counter.

However, when conditioning on a secret, we need to ensure that no information
can be leaked through differences in the number of bank conflicts between the
two branches {\em or} through differences in the number of bank conflicts in
one branch that might result from differing sets of threads taking that
branch (see \stefan{Example}).
%
The type system lacks the precision to quantify the number of bank conflicts
and so, at this point, it is necessary to use the Quantitative Hoare Logic:
\[
\infer[]
{\Sigma, \pc \vdash e : \mathsf{bool}_{\lpair{\seclevel_e}{\divergence_e}}\\
\Sigma, (\pc \ljoin \seclevel_e) \vdash s_1\\
\Sigma, (\pc \ljoin \seclevel_e) \vdash s_2\\
\{\Phi; 0; \{\}\} s_1 \sim s_1 \{\Phi'; 0; \{\}\}\\
\{\Phi; 0; \{\}\} s_2 \sim s_2 \{\Phi'; 0; \{\}\}\\
\{\Phi; 0; \{\}\} s_1 \sim s_2 \{\Phi'; 0; \{\}\}\\
}
{\Sigma, \pc \vdash \mathsf{if}~e~\mathsf{then}~s_1~\mathsf{else}~s_2}
\]
We require that the number of bank conflicts in both branches is
unaffected by the set of active threads by relating each branch to itself,
and we require that the two branches have the same number of bank conflicts
under any differences in secret information by relating them to each other.

We prove soundness of the IFC type system by showing that well-typed programs
are related to themselves with empty starting potential.
Composing this result with Theorem~\ref{thm:logic-sound} ensures that
two runs
of a well-typed program with low-equivalent starting configurations do not
differ in the number of bank conflicts.

\begin{theorem}[Soundness of Type System]
If~$\Sigma, \bot \vdash p$
and~$\Phi$ is any condition requiring that the two starting configurations
be low-equivalent,
then $\{\Phi, 0, \{\}\} p_1 \sim p_2 \{\Phi', 0, \{\}\}$.
\end{theorem}



% Explain the type system.
% Relational Hoare Logic
% Logical relation
