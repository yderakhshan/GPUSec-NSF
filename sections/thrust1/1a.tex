\subsubsection{Task 1.a.} \farzaneh{What we have developed so far with Godha.
With focus on bank conflicts, but general enough to all sorts of costs.}
This proposed tasks consists of two main contributions, first we will develop an information flow control (IFC) type system that ensures secure flow of information considering the timing attack.
%
This would be a static type system.
%
Next, we develop a resource-aware (?) Relational Hoare Logic as a semantic typing for IFC (?). 
%
\farzaneh{Put more here based on Amal and Jan's RHL.}

\paragraph{IFC type system.}

\paragraph{Resource-aware evaluation rules.}

\paragraph{Resource aware - Relational Hoare Logic.}
The judgment will be of the form 
%
\[\{\Phi; Q; X\} p_1 \sim p_2 \{\Phi'; Q'; X'\}\]
%
where $p_1$ and $p_2$ are programs the we want to relate.%
$\Phi$ is the precondition that describes the relation between the two states before running programs $p_1$ and $p_2$. 
%
$Q$ is
%
$X$ is
%
$\Phi'$ is the precondition that describe the relation between the two states after running programs $p_1$ and $p_2$.
%
$Q'$
%
$X'$

We will need two forms of rules, synchronized and unscynchronized:
\[
\infer[]{ \Phi \Rightarrow \mathtt{diff\_conflicts}(o_1,o_2) \le n\\  \Phi \vdash e_1 \sim e_2 : m}{\{\Phi; Q+n+m; X\} S_1[o_1] \leftarrow e_1 \sim S_2[o_2] \leftarrow e_2 \{\Phi; Q; X\}  }
\]

% Explain the type system.
% Relational Hoare Logic
% Logical relation