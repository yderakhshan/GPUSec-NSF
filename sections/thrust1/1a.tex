\subsubsection{Task 1.a. Information Flow Control Type System for GPU Timing Attacks}

This proposed task consists of two main contributions.
%
First we will develop an information flow control (IFC) type system that ensures noninterference, even up to timing attacks that observe aspects such as bank conflicts and shared memory accesses.
%
At key points, for example at a conditional where the condition depends on high-security information, the type system must ensure that the execution costs (e.g., bank conflicts, memory accesses) of the two branches are identical.
%
It does this by ``calling out'' to the second contribution of this task,
a resource-aware Relational Hoare Logic~\cite{Relcost} for quantitatively bounding the difference in execution cost between two evaluation of a statement or expression (e.g., that differ only on high-security inputs).
%

\farzaneh{Put more here based on Amal and Jan's RHL.}

\paragraph{IFC type system.}
Information flow control (IFC) type systems typically extend conventional type systems by tracking the security level of variables in addition to their data types.
%
In doing so, the type system can, for example, restrict assignments that would
cause high-security information to flow to a low-security variable.
%
As an example, our typing judgment for CUDA statements~$s$ is
$\Sigma, \pc \vdash s$,
where~$\Sigma$ is a context providing the types of constants and variables,
and $\pc$ is the security level of the program counter, which makes the type
system flow-sensitive by indicating whether the control flow of the program
has been influenced by high-security information.


\paragraph{Resource-aware evaluation rules.}

\paragraph{Resource aware - Relational Hoare Logic.}
The judgment will be of the form 
%
\[\{\Phi; Q; X\} p_1 \sim p_2 \{\Phi'; Q'; X'\}\]
%
where $p_1$ and $p_2$ are programs the we want to relate.%
$\Phi$ is the precondition that describes the relation between the two states before running programs $p_1$ and $p_2$. 
%
$Q$ is
%
$X$ is
%
$\Phi'$ is the precondition that describe the relation between the two states after running programs $p_1$ and $p_2$.
%
$Q'$
%
$X'$

We will need two forms of rules, synchronized and unscynchronized:
\[
\infer[]{ \Phi \Rightarrow \mathtt{diff\_conflicts}(o_1,o_2) \le n\\  \Phi \vdash e_1 \sim e_2 : m}{\{\Phi; Q+n+m; X\} S_1[o_1] \leftarrow e_1 \sim S_2[o_2] \leftarrow e_2 \{\Phi; Q; X\}  }
\]

% Explain the type system.
% Relational Hoare Logic
% Logical relation
