\subsubsection{Task 1.c}
In the third subtask of Thrust 1, we plan to implement the ideas of Tasks 1.a
and 1.b in a program analysis tool based on RaCUDA.
%
Because in CUDA, like C/C++, all variables are declared with their types, the
implementation of the type system is a relatively straightforward type checker,
with no inference necessary.
%
We will extend the syntax of the source language
to allow data types to be annotated with
security levels, so that programmers can indicate which variables and
function arguments should be considered high-security.~\footnote{As an
alternative, we could assign security levels to input channels, e.g.,
\texttt{stdin}, the command line, and different files, and propagate these
security levels to variables.}
%
The implementation challenge will be in implementing the relational
Hoare logic, which is used at certain points in the type system, on top of
the quantitative logic implemented by RaCUDA.

As a prelude to describing our extensions to implement the relational logic,
we give a bit of background on the design of RaCUDA.
%
RaCUDA is built on an earlier tool, Absynth, which accepts code in a simple
IMP-like imperative intermediate representation, annotated with \texttt{tick}
commands.
%
The statement~\texttt{tick}~$n$ for some integer~$n$ consumes~$n$ units of
potential ($n$ may be negative, indicating that potential becomes available).
%
Absynth analyzes the IMP program to produce sound upper bounds on the total
units of potential consumed by \texttt{tick}s on any run of a program.
%
It does this in three phases: it first performs an abstract interpretation
to collect facts about variables corresponding to the~$\Phi$ of the Hoare
logic rules.
%
Second, it applies the quantitative Hoare logic rules to build a set of
constraints on how potential is converted and used throughout the program.
%
Third, this set of constraints is solved using an off-the-shelf linear
program solver.
%
RaCUDA converts CUDA kernels to an extended version of Absynth's IMP
representation: it adds AST nodes to indicate shared and global memory
reads, which will correspond to different amounts of resource consumption
depending on the particular access patterns.
%
RaCUDA also extends Absynth's abstract interpretation pass to collect
CUDA-specific information.
%
During this pass, this information is used to convert the AST nodes
corresponding to memory accesses into \texttt{tick}s that indicate the
possible numbers of bank conflicts and global memory accesses (for example,
if the abstract interpretation has determined that a variable~\texttt{x} is
uniform across a warp, then~\texttt{S[x]} would not consume any resources).
%
The remainder of Absynth's analysis of the resource usage is unchanged.

The key to implementing a relational logic on top of RaCUDA is therefore
modifying the pass that assigns \texttt{tick}s to shared memory accesses
(and other resource-consuming operations once we extend our analysis beyond
bank conflicts).
%
Rather than assign a \texttt{tick} value corresponding to the maximum
number of bank conflicts, the code at this point should assign a value
corresponding to the \texttt{diff\_conflicts} function of our shared memory
rules in the relational logic.
%
The implementation of \texttt{diff\_conflicts} will take into account the
index expression and information about variables gained through abstract
interpretation, as well as additional information about the thread mask
which we will extend the abstract interpretation pass to track.
