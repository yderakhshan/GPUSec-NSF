\subsubsection{Task 1.b.}
In Task 1a we follow the classical approach of establishing noninterference to enforce security and ensure that no information is leaked to the attacker. This classical approach, however, is not always feasible or realistic. 
%
In a realistic setting, small leaks of information should be tolerated by the system as long as they are limited and do not significantly compromise the secrecy or identity of other parties.
% 
Even if the attacker observes a different number of bank conflicts in two runs of the program, they may not be able to deduce ``too much'' information from it. 
%
The precise threshold for “too much'' varies depending on the specifics of the system. 
%
For example, in the AES motivating example, if the encryption key rotation is set such that after every ten runs, a new random key is generated, then leaking one bit of information in every run cannot leak more than 10 bits of the ciphertext.
%
If we can quantify information leakage, we can compare it with the established threshold to ensure the system does not significantly compromise security.
%
In this task, we plan to maintain a realistic perspective on security by quantifying leaks via the timing side-channel.
%
We propose using the resource-aware Relational Hoare logic, along with its possible extensions, to manage quantitative information flow control based \stefan{insert ``on''?} min-entropy and min-capacity, as introduced by Smith~\cite{smith2011ICQE}.
%
The primary challenge for calculating the min-entropy is determining the cardinality of possible bank conflicts between two program executions that agree on low inputs.
%
To illustrate this approach, consider the following example, which uses a secret $h$ storing a two-bit value:
\begin{lstlisting}
    if (h=00) {
      S[2 * threadIdx] = 0;} 
    else {
      if (h=01) {
        S[threadIdx] = 0;}
      else {
        if (h=10) {
         S[2 * threadIdx] = 0;}
        else {
         S[threadIdx] = 0;}}}
\end{lstlisting}
% in Figure~\ref{fig:qtable}.
Based on the secret $h$, the program sends threads into four branches; in the first and third branches, there are no bank conflicts; in the second and fourth branches, there will be 16 bank conflicts. \stefan{Other way around? Also, while there will be 16 banks with conflicts, the way the resource metric measures the impact of conflicts is in terms of the maximum number of threads per warp accessing any bank (minus one)---this intuitively seems more directly correlated to access latency since it's the number of sequential accesses that would need to be performed---so the first and third branches would have a ``2-way'' bank conflict and the cost would be 1. So in this example the relational logic actually gives a tight bound on the cardinality of bank conflict possibilities :)}
%
% The condition of the branch is not divergent, meaning that all threads will execute the same branch, but depending on the value of the secret the branch can be different.
% %
We model the value of the secret $h$ as a random variable.
%
The attacker knows this \textit{a priori} probability but cannot 
observe the exact secret stored in $h$.
%
For simplicity, let us assume that the probability distribution of the secret is uniform,
meaning that each possible 2-bit secret value $s$ for $h$ has the same probability of $1/4$, i.e., $P(h=s)= 1/4$ for any $s\in S=\{00,01,10,11\}$.
%
%
Min-entropy provides the probability of the attacker guessing the correct secret in one try.
%
In this case, the probability is $1/2$ since there are only two possible outcomes for the number of bank conflicts, i.e., the cardinality of the possible outcomes for a number of bank conflicts is $2$.
%
This shows that the number of observations is key in calculating the min-entropy.
% Min-entropy provides a measure to quantify the amount of  uncertainty of the attacker (\cite{min-entropy}).
% %
% It is defined as 
% $H_\infty(h)=-\log_2\max_s P(h=s)$,
% and captures the probability that the attacker correctly guesses the secret in one try.
% %
% For example, before executing the program, the min-entropy is $H_\infty(h)=-\log_2\max_s P(h=s)= - \log_2 \max_s \{1/4,1/4,1/4,1/4\}= 2$.
% %
% We can interpret the min-entropy as the probability of the attacker guessing the correct secret is $1/2^{H_\infty(h)}$.
% %
% When the random variable has the uniform probability, one can rewrite a priori min-entropy as $H_\infty(h)=-\log_2\max_s P(h=s)=-\log_2 |S|$.
% %
% In this case, we can also interpret it as there are two bits that the attacker needs to know to fully guess the secret.
% %

% After executing the program, however, the attacker can observe the number of bank conflicts and learn information about the secret based on the number of observed bank conflicts.
% %
% If the attacker observes zero bank conflicts, then it can immediately rule out the secret having values $01$ and $11$.
% %
% While, if the attacker observes $16$ bank conflicts, then it can rule out the secret having values $00$ and $11$.
% %
% In both cases, the attacker can guess the correct secret in one try with $1/2$ chance.
% %

% Min-entropy can also be used to calculate a posteriori uncertainty of the attacker  (after the observation) using a conditional probability:
% $H_\infty(h\mid b)=-\log_2\Sigma_{o\in O} P(b=o) \max_s P(h=s|b=o)$,
% where $b$ is a random variable referring to the number of bank conflicts, $o$ is the number of bank conflict observed, and $O$ is the set of possible observations.
% %
% In our example, we know that the possible set of observations are $O=\{0,16\}$, the probabilities are $P(h=00,o=0)=P(h=01,o=16)=P(h=10,o=0)=P(h=11,o=16)=1$ with the rest of combinations having value $0$.
% %
% We can calculate the a posteriori entropy as 
% \[\begin{array}{ll}
%     H_\infty(h\mid b)=&-\log_2\Sigma_{o\in O} P(b=o) \max_s P(h=s|b=o)=\\ &-\log_2 (P(b=0)\max_s P(h=s|b=0) + P(b=16)\max_s P(h=s|b=16))=\\
%     &
%     -\log_2 (\max_s P(h=s,b=0) + \max_s P(h=s,b=16))=\\
%     & 
%     -\log_2 (1+1)= 1
% \end{array}\]
% We can interpret the a posteriori min-entropy as the probability of the attacker guessing the secret correctly after observing the number of bank conflicts is $1/2^{ H_\infty(h\mid b)}= 1/2$

% The amount of information leak can be quantified as $\mathsf{leak}= H_\infty(h) - H_\infty(h\mid b)$, which is equal to $2-1=1$ in our example.
% %
% This means that in the worst case scenario, the run will leak 2 bits of information.


% Smith~\cite{smith} showed that in the case where the program is deterministic and the probability of the secret is uniform, the min-entropy can be reduced to $\mathsf{leak}=\log_2 |O|,$ where $O$ is the set of all feasible obsevations. 
% %
% (for other probability distributions it is $\le \log_2 |O|.$)
% %
% This shows that the number of observations is key in calculating the min-entropy. 
% %
% $\{0,3\}$ will leak the same as $\{0,16\}$.

% Our proposed noninterference theorem in \ref{task:1a} ensures that no information will be leaked.
% %
% In other words, if we can prove the noninterference theorem for $s$, aka $\{\Phi;0; \{\}\}s \sim s\{\Phi;0;\{\}\}$, we know that all threads will have the same number of bank conflicts.
% %
% There is only one feasible observation and thus $\mathsf{leak}\le \log_2 |1|=0.$
% one of the columns is all $1$ and every other column has value $0$.
%
% Now let us see what it means if we have $\{\Phi;0; \{\}\}s \sim s\{\Phi;n;\{\}\}$.
% %
% It states that the differences between the number of bank conflicts in any two runs of the program is capped by $n$.
% %
% % All columns are the same except $n$ consecuritve columns that can be different.
% %
% This automatically provides an upper bound $n$ on the number of feasible observations, given that the number of bank conflicts is always a positive natural number. 
% %
% However, this upper bound in many cases is too generous.
% %
% For example, for program $s_0$ given above, we have $\{\Phi;0; \{\}\}s_0 \sim s_0\{\Phi;16;\{\}\}$, while we with 0 and 16.
% %

If we can prove the noninterference theorem discussed in Task 1a  for $p$, aka $\{\Phi;0; \{\}\}p \sim p\{\Phi;0;\{\}\}$, there is only one possible outcome for the number of bank conflicts across different executions, i.e., the cardinality of the set of bank conflicts is $1$.
%
Now consider the case where we have $\{\Phi;0; \{\}\}p \sim p\{\Phi;n;\{\}\}$ for $n > 0$.
%
It indicates that the differences between the number of bank conflicts in any two runs of the program is capped by $n$.
%
This automatically provides an upper bound $n$ on the number of possible observations, given that the number of bank conflicts is always a positive natural number. 
%
We plan to use this upper bound to calculate the min-entropy of a program.
%
However, this upper bound, in many cases, is too generous.
%
For example, for sample program $p_0$ given above, we have $\{\Phi;0; \{\}\}p_0 \sim p_0\{\Phi;16;\{\}\}$, giving us an upper bound of $16$ on the cardinality of possible outcomes, while the actual cardinality is only $2$.
%
We will investigate different interpretations of the potential function and design of a resource-aware Relational Hoare Logic that can provide a more accurate estimate of the cardinality of possible outcomes.
%
Furthermore, we will explore more general approaches to quantitative information flow control based on g-vulnerability rather than min-entropy~\cite{alvim2020springer}.
%
The g-vulnerability is defined using a function that quantifies the gain an attacker experiences by choosing from a set of its actions. 
%
% We propose the following adapted rule for the resource aware hoare logic to provide an upper-bound on the cardinality of the possible number of bank conflicts rather than the maximum number of bank conflicts.
%
% The set can (and should for decidability reasons) still be an over-approximation but it can reduce the space further.
%
% Here $\mathtt{card\_conflicts}(o)$ calculates an upper-bound on the cardinality of the set of all possible bank conflicts  for index $o$ while $m$ represents the cardinality of possible bank conflicts when executing the expression $e$.
% %
% Since we are dealing with cardinalities, the proper operation for composing the cardinalities is multiplication, rather than addition.
% \[
% \infer[]{\Phi[e/S[o]] \Rightarrow \mathtt{conflicts}(o_1) \le n\\  \Phi \vdash e : m}{\{\Phi; Q\times n \times m; X\} S[o] \leftarrow e  \{\Phi; Q; X\}  }
% \]

% The {\em potential function} $Q$ maps program states to rational numbers.
% %

% The standard way to quantify security is by considering the prior knowledge of the attacker about the secret as  a random variable of which the attacker only knows the distribution of, and then we model the program as a channel that maps each input state to an observable output by the attacker $\mathsf{C}: \mathcal{X} \rightarrow \mathcal{T}$. Here the observable output is the number of outcomes, which we right it as a $\mathcal{T}$. Here, we deal with deterministic programs, meaning that for each input state there is exactly one observable output (number of bank conflicts.)



% The g-vulnerability of the posterior and thus the overall leak depends on the characteristics of the program as a channel.
% %
% We propose  designing a new Hoare Logic by adapting the resource aware Hoare logic for CUDA to capture the characteristics of the program as a channel.



% After that, we plan to also consider arbitrary distributions other that uniform. 
% %
% We have to work with the original formula.
% %
% (no nondeterministic for now).
% %
% An unreliable environment might induce noise in the observations of the attacker making the attacker unable to distinguish the difference between certain results, as long as the difference is is not ``too much''. 
% Based on the noise of the environment and the abilities of the attacker, we can also say whether two possible observations are distinguishable to the attacker.
% %
% For example $1$ vs $2$ and $2$ vs $3$ is not observable to the attacker.
% %
% But $1$ vs $3$ is.
% \farzaneh{is it a problem that we don't have a transititivity? Add more here.}

% It is defined as as the negative logarithm of the probability of the most likely outcome.

% The guarantees we obtain are upper bounds
% on the amount of information about the input that an adversary can extract by
% observing which memory locations are present in the CPU's cache after execution
% of the program;




% We plan to incorporate quantitative analysis in the RHL, limiting the probability of the leak, e.g., via shared timing attacks, to a set threshold.

% \farzaneh{add a connection to the motivating example:In our motivating example: For each plaintext we get x bits of information about the key.  %
% If we run it multiple times with different plaintext, we can find the key compeltely.   A possible scenario: leaking three bits can be tolerated because after every ten runs a new random key is generated? this is called key rotation.}


  
    % Such theories offer an attractive way to relax the standard noninterference properties, letting us tolerate “small” leaks that are necessary in practice. 
%

% In this task, we plan to maintain a realistic perspective on security by allowing the possibility of leaks as long as they are limited and do not leak too much information. The precise threshold for “too much information” varies depending on the specifics of the system. We define the threshold by incorporating quantitative analysis in the RHL, limiting the probability of the leak, e.g., via shared timing attacks, to a set threshold.


    % In our motivating example: For each plaintext we get x bits of information about the key. 
    % %
    % If we run it multiple times with different plaintext, we can find the key compeltely : Quantitative aspect.
    % %
    % A possible scenario: leaking three bits can be tolerated because after every ten runs a new random key is generated? this is called key rotation.
    % Min entropy.
    % Such theories offer an attractive way to relax the standard noninterference properties, letting us tolerate “small” leaks that are necessary in practice. 


    % A quantitative metric measuring “distinguishability”
    % should account for an optimal guessing strategy employed by the
    % attacker. Such an optimal guessing strategy should guess the most
    % likely victim access pattern by leveraging full knowledge of the
    % obfuscating scheme's probabilistic properties


