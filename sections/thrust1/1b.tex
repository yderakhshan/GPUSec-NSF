\subsection{Task 1.b.} Quantitative aspect.

In our motivating example: For each plaintext we get x bits of information about the key. 
%
If we run it multiple times with different plaintext, we can find the key compeltely : Quantitative aspect.
%
A possible scenario: leaking three bits can be tolerated because after every ten runs a new random key is generated? this is called key rotation.
Min entropy.
Such theories offer an attractive way to relax the standard noninterference properties, letting us tolerate “small” leaks that are necessary in practice. 


A quantitative metric measuring “distinguishability”
should account for an optimal guessing strategy employed by the
attacker. Such an optimal guessing strategy should guess the most
likely victim access pattern by leveraging full knowledge of the
obfuscating scheme’s probabilistic properties


\begin{quote} 
    How can we quantify the leakage 
\end{quote}
